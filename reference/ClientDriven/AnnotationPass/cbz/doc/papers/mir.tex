The goal of the MIR level is to provide a well-defined internal
representation that is simpler than the AST representation while still
remaining machine-independent.  At the MIR level, the tree structure of the
AST has been flattened so that each basic block is a list of instructions,
and each instruction is considerably simpler, allowing only a single
assignment operator and only goto's as control flow.  In addition to the
canonical MIR format, there is also an intermediate MIR format that
simplifies control flow without simplifying expressions.

The {\em dismantler} is responsible for converting an AST into MIR format.
The dismantler can be invoked from the command line by specifying the 
\texttt{-dismantle} flag; to dismantle control flow only, use the
\texttt{-dismantle-control} flag.  

The dismantler can also be invoked programmatically, which can be useful for
restoring the MIR format after a transformation.  Note that the dismantler
has been designed to be idempotent, so that dismantling a portion of code
multiple times has the same effect as dismantling it once.\footnote{Within
the context of C-Breeze, it is important to note that dismantling a file,
outputting it to C code, and then re-dismantling the generated C file is
\emph{not} guaranteed to be idempotent.  We view the dismantler as idempotent
only when applied to ASTs.} To use the new dismantler programmatically, use
one of the static methods defined on the \texttt{Dismantle} class.  You must
include the file \texttt{dismantle.h} in order to see this class.  The method
\texttt{Dismantle\\::dismantle\_control()} will dismantle control flow only.
To completely dismantle the code, use the method
\texttt{Dismantle::dismantle()}.  Note that although the \texttt{Dismantle}
class is a \texttt{Changer}, it is \emph{not} intended to be used in a call
to \texttt{Node::change()}; using the class directly in this way can lead to
subtle bugs.

Finally, as mentioned in Section~\ref{sec:Basics}, the {\tt -cfg} flag 
instructs C-Breeze to both convert to MIR and create a CFG.

The MIR format is not well-suited for loop transformations on
multi-dimensional arrays, such as loop interchange or unroll and jam.  The
difficulty arises because in our MIR format each \texttt{operandNode} has an
optional array subscript at the end.  Since it only has a single subscript,
multi-dimensional arrays cannot be represented directly but instead require
the use of a temporary variable.



\subsection{\label{assume:ast} AST input}

The code that the dismantler accepts as input must be a tree, and not
a more general graph.  If the input code is not a tree, the output of
the dismantler is not guaranteed to be correct, reasonable, or
meaningful.  Two practices will ensure that the code the dismantler
receives is a tree.  First, use the dismantler as the first phase after
parsing the input file.  Second, use the
\texttt{ref\_clone\_changer::clone()} method whenever you are assigning a
node subclass pointer from a field in one object to a field in another
object.

The dismantler also assumes that it may destructively change any
\node\ in the input AST for use in the output AST.  This assumption
seriously affects what assumptions code using the dismantler can make
both before and after using the dismantler.  For example, any code
which relies on a particular instance of \node\ being the same before
and after dismantling is likely to have untold problems.  

% To help reduce the risk of these unintended consequences, any
% \texttt{Annotation}s in the input are \emph{guaranteed} to be invalid in
% the output.

% \footnote{The currently held view of \texttt{Annotation}s is that they are a
% bad idea.  Any information that an \texttt{Annotation} can contain would be
% best codified either in a field of a \node-subclass or in a user-maintained
% data structure that is passed from one phase to another.}


\subsection{\label{assume:hw} Hardware platform}

The MIR representation corresponds to the operations available on a modern
microprocessor.  There are three major assumptions underlying this
statement.

First, we assume that we are generating code for a load-store RISC-type
processor.  This assumption guides the selection of the basic addressable
memory location used in operations.  

Second, we assume that the architecture supports a register indirect with
offset addressing mode; if the base address of an array or structure is
loaded in a register, we can load the value from the proper offset location
in a single instruction.\footnote{Actually, this is a slight simplification.
For arrays, we allow the index to be either a constant, in which case the
previous statement is true, or a variable, in which case we require an
additional load of the variable's value and a multiply of the array element
size.  In this later case, we require the architecture to support a register
indirect with register offset addressing mode.}

Third, we assume that the instruction set architecture (ISA) supports
moving values from a condition register(s) into a general-purpose
register.  This assumption allows us to leave assignments where the
right-hand side uses a relational operator.  To determine the value of the
assignment, we can use shifting, masking, and inverting to extract the
relevant bit from the condition register value.

\subsection{\label{output} The MIR Format}

This section describes the details of the MIR format.  Three \node-subclasses
can appear in at the MIR level that cannot appear at the AST level:
\texttt{operandNode, conditiongotoNode,} and \texttt{threeAddrNode}.  We then
describe the restrictions that are placed on existing \node\ types that may
appear at the MIR level.


\subsubsection{\node-subclass prototypes}
\label{output:node} 

The \texttt{indexNode} class simply acts as a superclass for
\texttt{constNode} and \texttt{idNode}:

\begin{small}
\begin{verbatim}
class indexNode : public exprNode {
  indexNode(NodeType typ, typeNode * type, const Coord coord);
};

class constNode : public indexNode { ... };
class idNode : public indexNode { ... };
\end{verbatim}
\end{small}


The \texttt{conditiongotoNode} class corresponds to a conditional branch in
the ISA.  At the MIR level, all control flow structures in the input AST are
represented by combinations of \texttt{conditiongotoNode}s,
\texttt{gotoNode}s, and \texttt{labelNode}s.  The \texttt{Operator} in the
\texttt{op} field of the \texttt{conditiongotoNode} must be a relational
operator (eg: \texttt{<, <=, ==,} etc.).

\begin{small}
\begin{verbatim}
class conditiongotoNode : public gotoNode {
  conditiongotoNode(labelNode * label, indexNode * left, unsigned int op_id,
                    indexNode * right, const Coord coord = Coord::Unknown);
  void left(indexNode * left);
  indexNode * left(void);
  void right(indexNode * right);
  indexNode * right(void);
  void op(Operator * op);
  Operator * op(void);
};
\end{verbatim}
\end{small}

The \texttt{operandNode} class represents a memory location or address.

\begin{small}
\begin{verbatim}
class operandNode : public exprNode {
  operandNode(indexNode * the_var, const Coord coord = Coord::Unknown);
  operandNode(indexNode * the_var, bool star, bool addr, 
              const Coord coord = Coord::Unknown);
  operandNode(indexNode * the_var, bool star, bool addr, id_list * fields, 
              indexNode * array_index, const Coord coord = Coord::Unknown);

  void addr(bool addr) { _addr = addr; }
  bool addr() { return _addr; }
  void star(bool star) { _star = star; }
  bool star() { return _star; }
  void cast(typeNode * cast) { _cast = cast; }
  typeNode * cast(void) { return _cast; }
  const typeNode * cast(void) const { return _cast; }
  inline id_list & fields() { return _fields; }
  inline const id_list & fields() const { return _fields; }
  void index(indexNode * index) { _index = index; }
  indexNode * index(void) { return _index; }
  const indexNode * index(void) const { return _index; }
  void var(indexNode * var) { _var = var; }
  indexNode * var(void) { return _var; }
  const indexNode * var(void) const { return _var; }
};
\end{verbatim}
\end{small}

An \texttt{operandNode} represents a simple variable or constant when only
the \texttt{var} field is set.  Otherwise, it represents a memory location
(or the address of a memory location) that can be loaded in a single
instruction.  This latter case assumes that the base address of the
\texttt{var} field is already loaded into a register and the offset is either
constant or already computed.  In a grammatical sense, the
\texttt{operandNode} is defined by the following rules:

\begin{eqnarray*}
operand & \rightarrow & [typecast][\texttt{\&}] (\texttt{*} var) \{.field\}^*
          [ \texttt{[} index \texttt{]}] \\
        & \rightarrow & [typecast][\texttt{\&}] var \{.field\}^*
          [ \texttt{[} index \texttt{]}]
\end{eqnarray*}

The \texttt{threeAddrNode} is the quintessential form for all arithmetic
statements in the dismantler's output:  

\begin{small}
\begin{verbatim}
class threeAddrNode : public stmtNode {
 public:
  // a = b;
  threeAddrNode(operandNode * lhs, operandNode * rhs, 
                const Coord coord = Coord::Unknown);
  // a = -b;
  threeAddrNode(operandNode * lhs, unsigned int op_id, 
                operandNode * rhs, const Coord coord = Coord::Unknown);
  // a = sizeof(<type>)
  threeAddrNode(operandNode * lhs, typeNode * type,
                const Coord coord = Coord::Unknown);
  // a = b (op) c;
  threeAddrNode(operandNode * lhs, operandNode * rhs1, unsigned int op_id,
                operandNode * rhs2, const Coord coord = Coord::Unknown);
  // a = b(...);
  threeAddrNode(operandNode * lhs, operandNode * func, 
                operand_list * arg_list, const Coord coord = Coord::Unknown);
  // a(...);
  threeAddrNode(operandNode * func, operand_list * arg_list, 
                const Coord coord = Coord::Unknown);

  void lhs(operandNode * lhs);
  operandNode * lhs(void);
  void rhs1(operandNode * rhs1);
  operandNode * rhs1(void);
  void op(Operator * op);
  Operator * op(void);
  void rhs2(operandNode * rhs2);
  operandNode * rhs2(void);
  void sizeof_type(typeNode * type);
  typeNode * sizeof_type(void);
  inline operand_list & arg_list();
  inline const operand_list & arg_list() const;
};
\end{verbatim}
\end{small}

Using \texttt{threeAddrNode}, we can represent assignment of variables, unary
operations to operands, binary operations to operands, function calls, and
determining the \texttt{sizeof} a type or operand.  For function calls, the
assignment is optional.  The following grammar describes the forms that
a \texttt{threeAddrNode} can take.

\begin{eqnarray*}
threeAddrNode & \rightarrow & \texttt{lhs} = [unaryOperator]\ \texttt{rhs1} \\
              & \rightarrow & \texttt{lhs} = \texttt{rhs1}\ binaryOperator\ 
					     \texttt{rhs2} \\
              & \rightarrow & \texttt{lhs} = \texttt{sizeof(} type | 
						\texttt{rhs1} \texttt{)} \\
              & \rightarrow & [\texttt{lhs} =]\ \texttt{rhs1}(arglist) \\
arglist & \rightarrow & arg\ argrest \\
argrest & \rightarrow & , arg\ argrest \\
        & \rightarrow & \epsilon\\
arg     & \rightarrow & operand
\end{eqnarray*}

\subsubsection{Details of the MIR Format}
\label{output:output}

The MIR format transforms the AST in the following ways.  It converts the
initialization of automatic variables into assignment statements at the
beginning of the relevant block.  Next, all \texttt{labelNode}s are modified
so that the \texttt{stmt} field contains a \texttt{blockNode} containing only
an empty statement (\texttt{exprstmtNode} with a \texttt{NULL} expression).
Finally, the resulting AST is flattened to remove unnecessary nesting of
blocks, and all variable declarations are moved up to the \texttt{procNode}
level.

The first level of dismantling transforms complex control flow constructs.
During this phase, all loops (\texttt{whileNode, doNode, forNode}), if
statements (\texttt{ifNode}), and switch statements (\texttt{switchNode}) are
replaced with equivalent \texttt{conditiongotoNode}s, \texttt{gotoNode}s, and
\texttt{labelNode}s.  Additionally, the ternary operator
(\texttt{ternaryNode}) and short-circuit conditional operators (\texttt{\&\&,
||}) are transformed into equivalent \texttt{ifNode}s, which are subsequently
transformed as mentioned above.

The second dismantling level transforms expressions into a series of
\emph{simple expression} with intermediate results.  Roughly speaking, a
simple expression corresponds to an instruction of a modern RISC
microprocessor.  Thus, simple expressions contain at most one operator
(excluding assignments ) and one assignment; operands must be addressable by
a single \texttt{load} instruction.

Fully dismantled code only contains five types of statement nodes in the body
of a procedure: \texttt{labelNode, gotoNode, conditiongotoNode,
threeAddrNode,} and \texttt{returnNode}.  This means that for processing the
code only five methods for \texttt{Walker}s, \texttt{Visitor}s, and
\texttt{Changer}s are needed to visit these nodes: \texttt{at\_label(),
at\_goto(), at\_conditiongoto(), at\_threeAddr(),} and
\texttt{at\_return().}  The argument lists for these methods matches the
convention for the other \texttt{at\_*()} methods in the relevant class.

In MIR form, the \texttt{procNode} for the procedure has two new fields.  The
\texttt{return\_label()} methods access a field containing a pointer to the
\texttt{labelNode} at the beginning of the return block.  The
\texttt{declNode} for the returned value is accessible using the
\texttt{return\_decl()} methods.


\subsubsection{Tree Traversal of MIR Code}
\label{migrate:tree}

This section describes how to traverse MIR code.  

Traversal of \texttt{conditiongotoNode}:  Since \texttt{conditiongotoNode}
inherits from \texttt{gotoNode}, unless a traverser (\texttt{Visitor,
Walker,} or \texttt{Changer}) defines an \texttt{at\_conditiongoto()} method,
the traverser's \texttt{at\_goto()} method will be invoked on the
\texttt{conditiongotoNode}.  In addition to traversing the
\texttt{conditiongoto\-Node}, \texttt{Walker}s and \texttt{Changer}s will
traverse the \texttt{indexNode}s on the left and right of the conditional
expression (the \texttt{left()} and \texttt{right()} fields).

Traversal of \texttt{operandNode}:  Unless a traverser defines an
\texttt{at\_operand()} method, the traverser's \texttt{at\_expr()} method
will be invoked.  \texttt{Walker}s and \texttt{Changer}s will traverse the
\texttt{indexNode} for the \texttt{var()} field, the \texttt{idNode}s in the
\texttt{id\_list} in the \texttt{fields()} field, and the \texttt{indexNode}
in the \texttt{index()} field.

Traversal of \texttt{threeAddrNode}:  By default, traversing a
\texttt{threeAddrNode} will invoke \texttt{at\_stmt()} if a traverser does
not define an \texttt{at\_threeAddr()} method.  Traversing the
\texttt{threeAddrNode} will also traverse the \texttt{operandNode}s in the
\texttt{lhs(), rhs1(),} and \texttt{rhs2()} fields, assuming that the field
is non-\texttt{NULL}, the \texttt{typeNode} in the \texttt{sizeof\_type()}
field, when the operator is a \texttt{sizeof}, and the \texttt{operandNode}s
in the \texttt{operand\_list} in the \texttt{arg\_list} field, when the
operator is a \texttt{FUNC\_CALL}.
