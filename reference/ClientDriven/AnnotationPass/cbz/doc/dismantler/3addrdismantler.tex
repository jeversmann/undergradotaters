\documentclass{article}
\usepackage{epsfig, amssymb}
%\input{macros}

\footskip 35pt

\topmargin 0in
\advance \topmargin by -\headheight
\advance \topmargin by -\headsep
     
\textheight 9in
\oddsidemargin 0pt
\evensidemargin \oddsidemargin
\marginparwidth 0.5in
     
\textwidth 6.5in
\setlength{\parindent}{0ex}
\setlength{\parskip}{1ex}

%additional command definitions
\def\E{{\bf{\rm E}\/}}
\newcommand{\Real}{\mathrm{I\!R}}
\newcommand{\Tab}{\hspace{0.5cm}}
\newcommand{\norm}[1]{\left\|#1\right\|}
\newcommand{\normi}[1]{\left\|#1\right\|_\infty}
\newcommand{\abs}[1]{\left|#1\right|}
\newcommand{\QED}{$\Box$}
\newcommand{\node}{\texttt{Node}}

% use \> as a tab within the following definition
\newenvironment{algorithm}
{\ttfamily \begin{tabbing}
\hspace*{.35in}\=
\hspace*{.25in}\=
\hspace*{.25in}\=
\hspace*{.25in}\=
\hspace*{.25in}\=
\hspace*{.25in}\=
\hspace*{.25in}\=
\hspace*{.25in}\=
\hspace*{.25in}\=
\hspace*{.25in}\=\\
}
{\end{tabbing} \rmfamily}

\hyphenation{cond-i-tion-goto-Node}

\title{Design of the Three Address Dismantler for C-Breeze}
\author{Adam Brown, Samuel Z. Guyer, Calvin Lin}

\begin{document}
\maketitle

\section{Introduction}

The goal of the C-Breeze dismantler is to take a representation of a
source program and transform it into an equivalent, canonical
representation.  A rigidly specified canonical representation allows
subsequent compiler passes to make strong assumptions about the
abstract syntax tree (AST) that the pass will handle; these
assumptions allow a pass to handle fewer \node\ types, simplifying its
structure.  This document describes the C-Breeze dismantler including
its goals, its design, and its output.

This paper is organized as follows.  In Section \ref{prin}, we discuss
the design principles underlying the dismantler.  Section
\ref{assume} details assumptions about the underlying hardware and the
input AST.  The output from the dismantler, the canonical
representation, is presented in Section \ref{output}.  Section
\ref{limit} describes limitations on the optimizations and
transformations possible with the canonical representation.  Section
\ref{future} presents some ideas for future improvements to C-Breeze
made possible with this dismantler.

\section{\label{prin} Principles}

The design of the dismantler attempts to address several conflicting
goals.  One goal is to provide a canonical form of generated code be
to modern instruction set architectures (ISA).  A second goal is
modularity; the dismantler should be organized so that unrelated
transformations reside in separate code modules, thus allowing for
easier maintenance and subsequent improvement.  The final goal is that
the dismantler be funcationally idempotent; dismantling a portion of
code multiple times should have the same effect as dismantling it
once.  This third goal will be simplified by the regularity of the
generated output.

\subsection{\label{prin:close-to-hardware} Close to hardware}

The output from the dismantler should correspond to the basic
operations that a modern microprocessor can perform in a single
instruction.  The operands for an operation should correspond to a
single \texttt{load} instruction, assuming that the base address is
stored in a register.  Additionally, the destination for an operation
should correspond to a single \texttt{store}, also assuming that the
base address is stored in a register.  We discuss the particular
hardware assumptions in Section \ref{assume:hw}.

This decision also has implications for control flow.  Complex control
flow constructs, such a loops and short-circuit conditionals, should
be dismantled into simple \texttt{if}s where the condition is a
relational operation, the true branch is a \texttt{goto}, and the
false branch is empty.

\subsection{Orthogonality and modularization}

The dismantler performs several unrelated transformations to the input
code.  Since these transformations are logically orthogonal, they
should reside in different code modules.  By enforcing modularity, we
simplify dismantler maintenance and allow future improvements to the
dismantler to be applied easily.

\subsection{\label{prin:idem} Idempotent}

The dismantler should transform any valid program, represented by an
AST into a canonical representation.  This functionality implies that
if the AST representation, or portions of it, are already in the
canonical representation, the dismantler will not further modify the
AST, or the respective portions of it.  A slight simplification states
that if the dismantler is called twice in succession on a piece of
code, the second pass will not produce output different from its
input.\footnote{Within the context of C-Breeze, it is important to
note that dismantling a file, outputting it to C code, and then
re-dismantling the generated C file is \emph{not} guaranteed to be
idempotent.  We view the dismantler as idempotent only when applied to
ASTs.}

\subsection{Regularity}

The dismantler should output a canonical representation of the code.
We utilize the regularity of the canonical representation to ensure
that the dismantler is idempotent.  The canonical representation for
the dismantler's output is explicitly stated in Section
\ref{output}.

\section{\label{assume} Assumptions and Implications}

The dismantler makes two types of assumptions.  First, it assumes
certain features for the AST that it receives as input.  Second, the
code output by the dismantler presupposes a certain ISA, thus making
the code close to a three address format without explicit loads or
stores.

\subsection{\label{assume:ast} AST input}

The code that the dismantler accepts as input must be a tree and not a
more general graph.  If the input code is not a tree, the output of
the dismantler is not guaranteed to be correct, reasonable, or
meaningful.  Two practices will ensure that the input code will be a
tree.  First, the dismantler should be the first phase used after
parsing the input file.  Second, the
\texttt{ref\_clone\_changer::clone()} method should be used when
assigning to any field in any subclass of \node.

The dismantler also assumes that it may destructively change any
\node\ in the input AST for use in the output AST.  This assumption
seriously affects what assumptions code using the dismantler can make
both before and after using the dismantler.  For example, any code
which relies on a particular instance of \node\ being the same before
and after dismantling is likely to have untold problems.  To help
reduce the risk of these unintended consequences, any
\texttt{Annotation}s in the input are \emph{guaranteed} to be invalid
in the output.\footnote{The currently held view of
\texttt{Annotation}s is that they are a bad idea.  Any information
that an \texttt{Annotation} can contain would be best codified either
in a field of a \node-subclass or in a user-maintained data structure
that is passed from one phase to another.}

\subsection{\label{assume:hw} Hardware platform}

In Section \ref{prin:close-to-hardware}, we mentioned that the code
generated by the dismantler corresponds to the operations available on
a modern microprocessor.  There are three major assumptions underlying
this statement.

First, we assume that we are generating code for a load-store
RISC-type processor.  This assumption guides the selection of the
basic addressable memory location used in operations.  The material
about the \texttt{operandNode} in Section \ref{output:node} provides
details.

Second, we assume that the architecture supports a register indirect
with offset addressing mode; if the base address of an array or
structure is loaded in a register, we can load the value from the
proper offset location in a single instruction.\footnote{Actually,
this is a slight simplification.  For arrays, we allow the index to be
either a constant, in which case the previous statement is true, or a
variable, in which case we require an additional load of the
variable's value and a multiply of the array element size.  In this
latter case, we require the architecture to support a register indirect
with register offset addressing mode.}

Third, we assume that the instruction set architecture (ISA) supports
moving values from a condition register into a general-purpose
register.  This assumption allows us to leave assignments where the
right-hand side uses a relational operator.\footnote{Section
\ref{future:assign-relate} for more details.}  To determine the value
of the assignment, we can use shifting, masking, and inverting to
extract the relevant bit from the condition register value.

\section{\label{output} Dismantler output and \node-subclass prototypes}

In this section, we describe the form of the code generated by the
dismantler.  We introduce three \node-subclasses that appear in
dismantled code: \texttt{operandNode, conditiongotoNode,} and 
\texttt{threeAddrNode}.  Additionally, we introduce \texttt{indexNode}
which serves as a superclass for some existing \node\ types.  We then
describe the restrictions that are placed on existing \node\ types
that may appear in dismantled code.

\subsection{\label{output:node} \node-subclass prototypes}

The dismantler introduces four subtypes of \node.  The rigidity
provided by these \node\ types assists in ensuring that the dismantler
is idempotent.  The first type we examine is \texttt{indexNode}.

\begin{verbatim}
class indexNode : public exprNode {
  indexNode(NodeType typ, typeNode * type, const Coord coord);
};

class constNode : public indexNode { ... };
class idNode : public indexNode { ... };
\end{verbatim}

\texttt{indexNode} simply acts as a superclass for \texttt{constNode}
and \texttt{idNode}, which previously inherited from
\texttt{exprNode}.  This superclass is necessary because some of the 
other nodes output by the dismantler have fields that can contain either a 
\texttt{constNode} or a \texttt{idNode};  \texttt{indexNode} simplifies
the types of these fields.

The second type we discuss is \texttt{conditiongotoNode}.

\begin{verbatim}
class conditiongotoNode : public gotoNode {
  conditiongotoNode(labelNode * label, indexNode * left, unsigned int op_id,
                    indexNode * right, const Coord coord = Coord::Unknown);
  void left(indexNode * left);
  indexNode * left(void);
  void right(indexNode * right);
  indexNode * right(void);
  void op(Operator * op);
  Operator * op(void);
};
\end{verbatim}

\texttt{conditiongotoNode} corresponds to a conditional branch in the ISA.
All control flow structures in the input AST are transformed into
combinations of \texttt{conditiongotoNode}s, \texttt{gotoNode}s, and
\texttt{labelNode}s by the dismantler.  The \texttt{Operator} in the
\texttt{op} field of the \texttt{conditiongotoNode} must be a relational
operator (eg: \texttt{<, <=, ==,} etc.).

The third type of \node\ is \texttt{operandNode}.

\begin{verbatim}
class operandNode : public exprNode {
  operandNode(indexNode * the_var, const Coord coord = Coord::Unknown);
  operandNode(indexNode * the_var, bool star, bool addr, 
              const Coord coord = Coord::Unknown);
  operandNode(indexNode * the_var, bool star, bool addr, id_list * fields, 
              indexNode * array_index, const Coord coord = Coord::Unknown);

  void addr(bool addr) { _addr = addr; }
  bool addr() { return _addr; }
  void star(bool star) { _star = star; }
  bool star() { return _star; }
  void cast(typeNode * cast) { _cast = cast; }
  typeNode * cast(void) { return _cast; }
  const typeNode * cast(void) const { return _cast; }
  typeNode * noncast_type(bool convertArrays) const;
  inline id_list & fields() { return _fields; }
  inline const id_list & fields() const { return _fields; }
  void index(indexNode * index) { _index = index; }
  indexNode * index(void) { return _index; }
  const indexNode * index(void) const { return _index; }
  void var(indexNode * var) { _var = var; }
  indexNode * var(void) { return _var; }
  const indexNode * var(void) const { return _var; }
};
\end{verbatim}

The role of \texttt{operandNode} is to represent a memory location or
address.  An \texttt{operandNode} represents a simple variable or
constant when only the \texttt{var} field is set.  Otherwise, it
represents a memory location (or the address of a memory location)
that can be loaded in a single instruction.  This latter case assumes
that the base address of the \texttt{var} field is already loaded into
a register and the offset is either constant or already computed.  In
a grammatical sense, we can view an \texttt{operandNode} by the
following rules.

\begin{eqnarray*}
operand & \rightarrow & [typecast][\texttt{\&}] (\texttt{*} var) \{.field\}^*
          [ \texttt{[} index \texttt{]}] \\
        & \rightarrow & [typecast][\texttt{\&}] var \{.field\}^*
          [ \texttt{[} index \texttt{]}]
\end{eqnarray*}

The fourth type of \node\ is the \texttt{threeAddrNode}.

\begin{verbatim}
class threeAddrNode : public stmtNode {
 public:
  // a = b;
  threeAddrNode(operandNode * lhs, operandNode * rhs, 
                const Coord coord = Coord::Unknown);
  // a = -b;
  threeAddrNode(operandNode * lhs, unsigned int op_id, 
                operandNode * rhs, const Coord coord = Coord::Unknown);
  // a = sizeof(<type>)
  threeAddrNode(operandNode * lhs, typeNode * type,
                const Coord coord = Coord::Unknown);
  // a = b (op) c;
  threeAddrNode(operandNode * lhs, operandNode * rhs1, unsigned int op_id,
                operandNode * rhs2, const Coord coord = Coord::Unknown);
  // a = b(...);
  threeAddrNode(operandNode * lhs, operandNode * func, 
                operand_list * arg_list, const Coord coord = Coord::Unknown);
  // a(...);
  threeAddrNode(operandNode * func, operand_list * arg_list, 
                const Coord coord = Coord::Unknown);

  void lhs(operandNode * lhs);
  operandNode * lhs(void);
  void rhs1(operandNode * rhs1);
  operandNode * rhs1(void);
  void op(Operator * op);
  Operator * op(void);
  void rhs2(operandNode * rhs2);
  operandNode * rhs2(void);
  void sizeof_type(typeNode * type);
  typeNode * sizeof_type(void);
  inline operand_list & arg_list();
  inline const operand_list & arg_list() const;
};
\end{verbatim}

The \texttt{threeAddrNode} is the quintessential form for all
arithmetic statements in the dismantler's output.  Using
\texttt{threeAddrNode}, we can represent assignment of variables,
unary operations to operands, binary operations to operands, function
calls, and determining the \texttt{sizeof} a type.  For function
calls, the assignment is optional.  The following grammar rules
describe the forms that a \texttt{threeAddrNode} can take.

\begin{eqnarray*}
threeAddrNode & \rightarrow & \texttt{lhs} = [unaryOperator]\ \texttt{rhs1} \\
              & \rightarrow & \texttt{lhs} = \texttt{rhs1}\ binaryOperator\ 
					     \texttt{rhs2} \\
              & \rightarrow & \texttt{lhs} = \texttt{sizeof(}type \texttt{)} \\
              & \rightarrow & [\texttt{lhs} =]\ \texttt{rhs1}(arglist) \\
arglist & \rightarrow & arg\ argrest \\
argrest & \rightarrow & , arg\ argrest \\
        & \rightarrow & \epsilon\\
arg     & \rightarrow & operand
\end{eqnarray*}

\subsection{\label{output:output} Dismantler output}

The dismantler provides two distinct levels of dismantled output.  At
all levels of dismantling, some common transformations are applied to
the input AST.  First, initialization of automatic variables is
removed from the declaration and made into an assignment statement at
the beginning of the relevant block.  Second, all \texttt{labelNode}s
are modified so that the \texttt{stmt} field contains a
\texttt{blockNode} containing only an empty statement
(\texttt{exprstmtNode} with a \texttt{NULL} expression).  Last, after
dismantling has occurred, the resulting AST is flattened to remove
unnecessary nesting of blocks, and all variable declarations are moved
up to the
\texttt{procNode} level.

The first level of dismantling transforms complex control flow
constructs.  During this phase, all loops (\texttt{whileNode, doNode,
forNode}), if statements (\texttt{ifNode}), and switch statements
(\texttt{switchNode}) are replaced with equivalent
\texttt{conditiongotoNode}s, \texttt{gotoNode}s, and
\texttt{labelNode}s.  Additionally, the ternary operator
(\texttt{ternaryNode}) and short-circuit conditional operators
(\texttt{\&\&, ||}) are transformed into equivalent \texttt{ifNode}s,
which are subsequently transformed as mentioned above.

The final dismantling level transforms any \emph{complex expression}
into a \emph{simple expression} or a series of simple expressions with
intermediate results.  A complex expression is any expression that
does not conform to the guidelines of Section
\ref{prin:close-to-hardware}.  An expression may be \emph{complex} for
several reasons.  Obviously, if the expression contains more than one
operation (binary or unary), it is complex.  For assignments, one of
the source operands may not be addressable by a single \texttt{load}
instruction; conversely, the destination operand may not be
addressable by a single \texttt{store} instruction.

\section{\label{limit} Limitations}

\subsection{No multi-dimensional arrays}

From Section \ref{output:node}, we know that each \texttt{operandNode}
has an optional array subscript at the end.  Since it only has a
single subscript we cannot directly represent multi-dimensional
arrays; they must be represented using a temporary variable.  For this
reason, loop transformations on multi-dimensional arrays, such as loop
interchange or unroll and jam, are unlikely to be implemented using
this IR.

\section{\label{future} Future directions/work}

\subsection{\label{future:assign-relate} Assignment of relational operators}

The current backend for PowerPC and x86 causes code similar to 
\verb|a = b < c;|, where \texttt{<} can be any relational operator, to
generate the code sequence
\begin{verbatim}
if ( b < c )
  a = 1;
else
  a = 0;
\end{verbatim}

However, the dismantler assumes that the underlying hardware
platform provides a mechanism for moving the value from the condition
register to a general purpose register.  Under this assumption, we
simply need to extract the relevant bit from the condition code after
the comparison.  We then assign the value of that bit to the variable.
For example, the code \verb|a = b < c;| from above can be transformed
into the straight line code in pseudo-PPC assembly
\begin{verbatim}
cmp b, c
mfcr a ; move from condition register
rs a, a, LT_BIT_POSITION ; right shift
andi a, a, 0x01 ; and immediate
\end{verbatim}

\subsection{\label{future:cfg} Redesigned control-flow graphs}

With the canonical intermediate representation, we may want to change the
CFG representation of code.  Primarily, \texttt{basicblockNode} should
be changed to include an exit condition and references/pointers to the
\texttt{basicblockNode}s corresponding to the exit condition being true
and false, respectively.  In this way, we can completely remove
\texttt{labelNode}s, \texttt{gotoNode}s, and \texttt{conditiongotoNode}s
from the CFG representation.  This new representation of the CFG will
allow loop preheaders to be implemented and allow better code placement,
see Section \ref{future:code}.

\subsection{\label{future:code} Code placement}

Once the \texttt{labelNode}s, \texttt{gotoNode}s, and 
\texttt{conditiongotoNode}s have been removed from the CFG representation,
we will be able to rearrange the ordering of \texttt{basicblockNode}s
more easily.  Currently, the control flow statements are still present
in the CFG representation explicitly;  their presence requires any code
attempting to reorder the basic blocks to search the body of affected
\texttt{basicblockNode}s.  This additional logic can be removed using
the proposed CFG form because control flow information is present directly
in the \texttt{basicblockNode}.

\subsection{\label{future:switch} Smarter \texttt{switch} statements}

The current dismantler changes \texttt{switch} statements into a series of
\texttt{if/then/else} statements.  With a large number of \texttt{case}
statements, this organization becomes increasingly inefficient.  We may
want to change the dismantler to generate jump tables indexed by the 
value in the \texttt{case} statements for large \texttt{switch} statements.

\appendix

\section{\label{migrate} Migration from Original Dismantler}

In this appendix, we describe the types of changes existing code needs
in order to use the AST output by the new dismantler.  We only address
dismantler output up to and including dismantling complex expressions.
We intended for this level of dismantling to be the most commonly used.
In Section \ref{migrate:other}, we briefly discuss how the other levels
of dismantled code differ from dismantling complex expressions.

\subsection{\label{migrate:using} Using the new dismantler}

There are two ways to use the new dismantler with your code: from the
command line, or programmatically.  In order to invoke the new dismantler
from the command line, you use one of the variants of the
\texttt{-dismantle*} flags.  To dismantle control flow only,
use \texttt{-dismantle-control}.  To dismantle through complex
expressions, use \texttt{-dismantle}.

To use the new dismantler programmatically, you should use one of the
static methods defined on the \texttt{Dismantle} class.  You must
include the file \texttt{dismantle.h} in order to see this class.  The
method \texttt{Dismantle\\::dismantle\_control()} will dismantle the
\texttt{unitNode} for control flow.  To completely dismantle
through complex expressions, use the method
\texttt{Dismantle::dismantle()}.  Although the \texttt{Dismantle} 
class is a \texttt{Changer}, it is \emph{not} intended to be used in a
call to \texttt{Node::change()}; using the class directly in this way
can lead to subtle bugs.

\subsection{\label{migrate:methods} Relevant methods}

Fully dismantled code contains five types of statement nodes in
the body of a procedure: \texttt{labelNode, gotoNode,
conditiongotoNode, threeAddrNode,} and \texttt{returnNode}.  This
means that for processing the code only five methods for
\texttt{Walker}s, \texttt{Visitor}s, and \texttt{Changer}s are needed
to visit these nodes: \texttt{at\_label(), at\_goto(),
at\_conditiongoto(), at\_threeAddr(),} and
\texttt{at\_return().}  The argument lists for these methods matches
the convention for the other \texttt{at\_*()} methods in the relevant
class.

Code that utilizes \texttt{at\_label()} or \texttt{at\_return()}
does \emph{not} need to change these methods.  The form of
\texttt{returnNode}s and \texttt{labelNode}s is the same in the new
dismantler as the old dismantler.  Both output code wherein a
\texttt{labelNode} contains a \texttt{blockNode} whose only statement is an
empty \texttt{exprstmtNode}.  Similarly, the expression in a
\texttt{returnNode} is an \texttt{idNode} or \texttt{NULL}.

Previously, code that wanted to insert a new \texttt{return} or insert
code to be executed before a return from the procedure had to find the
single \texttt{returnNode} in the dismantled output.  In finding this
\texttt{returnNode}, the code also found the variable used for the return 
value and the \texttt{labelNode} for the return block.  In code from
the new dismantler, the \texttt{procNode} for the procedure has two
new fields.  The \texttt{return\_label()} methods access a field
containing a pointer to the \texttt{labelNode} at the beginning of the 
return block.  The \texttt{declNode} for the returned value is accessible
using the \texttt{return\_decl()} methods.

\subsection{\label{migrate:tree} Tree traversal of nodes}

This section describes the tree traversal of the nodes in dismantled code.
As mentioned in Section \ref{migrate:methods}, \texttt{returnNode}s and
\texttt{labelNode}s are the same using either dismantler, and so their
tree traversal will be the same in either case.  Similarly, the traversal
of a \texttt{gotoNode} remains the same.

We first discuss the traversal of \texttt{conditiongotoNode}.  Since
\texttt{conditiongotoNode} inherits from \texttt{gotoNode}, unless a
traverser (\texttt{Visitor, Walker,} or \texttt{Changer}) defines an 
\texttt{at\_conditiongoto()} method, the traverser's 
\texttt{at\_goto()} method will be invoked on the 
\texttt{conditiongotoNode}.  In addition to traversing the 
\texttt{conditiongoto\-Node}, \texttt{Walker}s and \texttt{Changer}s will
traverse the \texttt{indexNode}s on the left and right of the conditional
expression (the \texttt{left()} and \texttt{right()} fields).

Next, we examine the traversal of \texttt{operandNode}.  Unless a traverser
defines an \texttt{at\_operand()} method, the traverser's
\texttt{at\_expr()} method will be invoked.  \texttt{Walker}s and
\texttt{Changer}s will traverse the \texttt{indexNode} for the \texttt{var()}
field, the \texttt{idNode}s in the \texttt{id\_list} in the
\texttt{fields()} field, and the \texttt{indexNode} in the 
\texttt{index()} field.

Last, we describe the traversal of \texttt{threeAddrNode}.  By default,
traversing a \texttt{threeAddrNode} will invoke \texttt{at\_stmt()}
if a traverser does not define an \texttt{at\_threeAddr()} method.
Traversing the \texttt{threeAddrNode} will also traverse the 
\texttt{operandNode}s in the \texttt{lhs(), rhs1(),} and \texttt{rhs2()} 
fields, assuming that the field is non-\texttt{NULL}, the
\texttt{typeNode} in the \texttt{sizeof\_type()} field, when the operator
is a \texttt{sizeof}, and the \texttt{operandNode}s in the
\texttt{operand\_list} in the \texttt{arg\_list} field, when the
operator is a \texttt{FUNC\_CALL}.

\subsection{\label{migrate:change} Changes to existing code}

This section discusses how common idioms in existing code map to the
new AST format.  In particular, we discuss locating assignments, function
calls, and binary and unary operators.  Lastly, we discuss the impact of
\texttt{conditiongotoNode} on code that handles control flow.

In code generated by the old dismantler, an assignment expression was
represented by a \texttt{binaryNode} whose \texttt{op()} had an
\texttt{id()} equal to \texttt{`='}.  This required code searching for
assignments to define an \texttt{at\_binary()} method that would check to
see if the \texttt{op()} of the node was an assignment.  In code
generated by the new dismantler, every \texttt{threeAddrNode} whose
\texttt{lhs()} field is non-\texttt{NULL} is implicitly an assignment.
Now, it suffices to define an \texttt{at\_threeAddr()} that checks to
see if the \texttt{lhs()} is non-\texttt{NULL}.

Function calls were represented by \texttt{callNode}s in the AST generated
by the old dismantler.  With the new dismantler, we have added the
\texttt{FUNC\_CALL} operator type.  Now, function calls are represented by
a \texttt{threeAddrNode} whose \texttt{op()} field has an \texttt{id()}
equal to \texttt{FUNC\_CALL}.  The \texttt{rhs1()} field contains the actual
function name or pointer, and the \texttt{arg\_list()} field contains
the list of \texttt{operandNode}s passed as arguments to the function.  As
mentioned above, if the \texttt{lhs()} field is non-\texttt{NULL}, then the
return value from the function call is assigned to the variable in that
field.

Previously, to locate binary or unary operators, a traverser needed to define
\texttt{at\_binary()} or \texttt{at\_unary()} methods, respectively.
Additionally, the \texttt{at\_binary()} method needed to check if the binary
operator was an assignment.  With \texttt{threeAddrNode}s, if the 
\texttt{rhs2()} field is non-\texttt{NULL}, then the node represents a binary
operation (and assignment); if the \texttt{rhs1()} field is
non-\texttt{NULL} and the operators \texttt{id()} is non-\texttt{NULL}
and is not \texttt{FUNC\_CALL}, then the node represents a unary
operation (and assignment).  One complication arises with the unary
\texttt{sizeof} operator.  If the \texttt{op()} field of the threeAddrNode
is \texttt{SIZEOF}, then the \texttt{sizeof\_type()} field will be
non-\texttt{NULL}, indicating that we are taking the \texttt{sizeof}
that type.  If the original source program was taking the \texttt{sizeof}
an expression, we now take the \texttt{sizeof} the type of that expression.

As we mentioned in Section \ref{migrate:methods}, the default behavior at
\texttt{conditiongotoNode} is to invoke \texttt{at\_goto()}.  Code
that is concerned with handling control flow should define an
\texttt{at\_conditiongoto()} method, because these two types of \node\ are
different enough that the logic \texttt{at\_goto()} uses is likely
insufficient to handle \texttt{conditiongotoNode}s.  Most likely, the
code in the traverser's \texttt{at\_if()} method should be adapted and moved
into an \texttt{at\_conditiongoto()} method.

\subsection{\label{migrate:other} Differences among levels of dismantled code}

In Section \ref{output:output}, we mentioned that there is one level
of dismantling other than dismantling complex expressions.  If this
other level is used, then additional \node\ types may still appear in
the AST output by the dismantler.  This section provides minimal
detail about this other level, because the intent of the
dismantler is to complete dismantling through complex expressions.

In code only dismantled through control flow, all \texttt{exprstmtNode}s
are the same as they were in the input AST except for those containing 
\texttt{ternaryNode}s and \texttt{binaryNode}s whose operator is 
\texttt{\&\&} and \texttt{||}.  All \texttt{loopNode}s and 
\texttt{selectionNode}s are replaced with equivalent sequences of statements
using \texttt{conditiongotoNode}s, \texttt{gotoNode}s and 
\texttt{labelNode}s.  For details, look at the source file
\texttt{src/helpers/dismantle-control.cc}.

It should be noted that for this level of dismantling, neither
\texttt{threeAddrNode}s nor \texttt{operandNode}s will appear in the
generated AST.

\end{document}
