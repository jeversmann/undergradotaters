\chapter{Limitations}\label{sec:limitation}
With our ported Broadway annotation language, there are certainly a lot of simplifications we made compared to the full language that the original code supported. Our ProcedureAnnotationInfo class actually can support almost all of the original Broadway's annotation grammar. The largest missing factor is the arbitrary C-code, since we do not allow C-Breeze to compile it. This essentially includes being able to use a C-block to include header files for parsers. The ProcedureAnnotationInfo class is easily extensible, and currently only provides an interface for features that we used for the AnnotationAnalysis or requested by the other group. 

Although our AnnotationAnalysis does perform error-checking, we only support a subset of the actual annotation grammar due to the time it took to port the code over and some difficulties with extracting information that was not preprocessed by C-Breeze. For example, we do not support nesting of conditions, because it be represented by using a hierarchical enum property with more values. in addition, we did not support the other type of property, set properties. The AnnotationAnalysis class does not handle ``new" or return ``memory blocs", because the current structures from the IDFA does not support them. This is something that that group wishes to address in the future.

Another major drawback is that we do not have the full access to the data-flow analysis as the original Broadway had.  Ironically, this is why decoupling dependences among major components of Broadway was difficult. Since we developed and ported the annotation separately from the group that handle the pointer analysis, there is not a complete interweaving that the original Broadway had. This prevented us from doing an efficient flow-sensitive analysis, because we would need to recompute the reaching defs. The proper approach to handle this would to have their system generate ``dummy" definitions after each library cool, where we can learn about the reaching definitions from library call to library call. Therefore, using information that the IDFA has already computed, rather than replicating work.

\chapter{Future}\label{sec:future}
There is certainly a significant amount of work remaining to completely port the original Broadway to LLVM. If given the opportunity, the best approach to  completely port over the annotations is to rewrite it in a modular fashion. The problem with the original Broadway's code is its dependencies on C-Breeze data structures for logic, data, and analysis. It was more like a limb of C-Breeze rather than a layer on top of it. If we were to extend on our current progress, we would certainly make a large effort to get arbitrary C-code to compile with LLVM's front-end. The main issue would be modifying the grammar  to use LLVM structures. The following step would then be to continue expanding ProcedureAnnotationInfo and AnnotationAnalysis with more logic to process the annotations information. Thus, transforming the old Broadway code into a API.